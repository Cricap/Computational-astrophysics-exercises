\documentclass[]{article}

%opening
\title{Manco il titolo mi ricordo}
\author{Cristina Caprioglio}
\date{2023/2024 }
	\usepackage{amssymb}
	\usepackage{amsmath}
	\usepackage{multicol}
	\usepackage{graphicx}
	\usepackage{physics}
\begin{document}

\maketitle

\section{Introduction}
Scopo+descrizione sistema
\section{Hydrostatic configuration}
\subsection{Physical model}
The first part of the project is about understanding how the gas is distributed in a cluster, so to do a proper gravitational model is needed.
We'll assume spherical symmetry, which will lead to all physical quantities depending only on $r$, the radial coordinate, which indicates the distance from the centre of the system.
The gas is supported by thermal pressure and it's described by the hydrostatic equilibrium equation:
\begin{equation}\label{idro_eq}
	\derivative{P}{r}=-\frac{GM(<r)}{r^{2}}\rho_{g}(r),
\end{equation}
which is a I order ODE. We have that in \eqref{idro_eq} $M(<r)$ indicates the total mass contained in a radius $r$, not just the mass of the gas, while $P$ and $\rho_{g}$ refer to the pressure and the density of the gas. \\
We have that the ICM is in equilibrium in the gravitational potential of the cluster, which is dominated by dark matter. \\
For the density profile of the latter we will assume a $Navarro-Frenk-White$ (NFW) $profile$, which takes the following form:
\begin{equation}\label{NFW}
	\rho_{DM}(r)=\frac{\rho_{DM,0}}{\left(\frac{r}{r_{s}}\right)\left(1+\frac{r}{r_{s}}\right)^{2}},
\end{equation}
where $\rho_{DM,0}$ and $r_{s}$ are parameters which depend on the cluster mass. 
Integrating \eqref{NFW} we can calculate the mass profile, which as an analytical solution:
\begin{equation}\label{massDM}
	M_{DM}(r)=\int_{0}^{r}4\pi r'^{2}\rho_{DM}(r')\dd{r'}=4\pi\rho_{DM,0}r_{s}^{3}\left[\ln\left(1+\frac{r}{r_{s}}\right)-\frac{r/r_{s}}{1+r/r_{s}}\right].
\end{equation}
To have a more realistic model we will also need to consider the presence of the BCG. Its stellar mass is described by the $Hernquist\; profile$:
\begin{equation}\label{Hern}
	M_{*}(r)=M_{BCG}\frac{r^{2}}{(r+a)^{2}},
\end{equation}
where $M_{BCG}$ is the mass of the elliptical central galaxy and $a$ is a scale related to the half-mass radius $r_{1/2}$: $r_{1/2}=(1+\sqrt{2})a$.\\
Assuming the ICM is described by the perfect gas equation of state (i.e. $P=\frac{k_{B}\rho_{g}T_{g}}{\mu m_{p}}$), \eqref{idro_eq} becomes:
\begin{equation}\label{hydroperf}
	\derivative{\ln \rho_{g}}{r}=-\frac{GM(<r)}{r^{2}}\frac{\mu m_{p}}{k_{B}T_{g}(r)}-\derivative{\ln T_{g}}{r},
\end{equation}
where $T_{g}(r)$ is the gas temperature, while $\mu ,\, m_{p},$ and $k_{B}$ are constants that represent the mean molecular weight, the proton mass and the Boltzmann constant respectively.\\
In the case of an isothermal gas and in the absence of the BCG eq.\eqref{hydroperf} has an analytical solution:
%\begin{equation}\label{isogas}
	\begin{gather}
		\rho_{g}=\rho_{0}\exp\left\{-\frac{27}{2}b\left[1-\frac{\ln (1+r/r_{s})}{r/r_{s}}\right]\right\}=\rho_{0}e^{-27b/2}\left(1+\frac{r}{r_{s}}\right)^{27b/(2r/r_{s})},\\
	with \quad b=\frac{8\pi G\mu m_{p}\rho_{DM,0}r_{s}^{2}}{27k_{B}T_{g,iso}}\notag
	\end{gather}
%\end{equation}
\subsection{The simulation}
\subsection{Results and conclusions}
\end{document}